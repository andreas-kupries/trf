% -*- latex -*-
\documentstyle {report}
\def\strong#1{{\bf #1}}
\def\lang#1{{\tt #1}}
\def\file#1{{\tt #1}}
\def\arg#1{{\tt $<$#1$>$}}
\def\trf{\strong {trf}}
\def\blob{\strong {blob}}
\def\blobx{\strong {blobX}}
\def\tcl{\lang {tcl}}
\def\cmd#1{{\tt #1}}
\def\opt#1{{\tt #1}}
\def\man#1{{\tt #1}}
\def\lib#1{{\tt #1}}
\def\url#1{{\tt #1}}

\def\version{1.0}
\begin {document}
\tableofcontents

\chapter {Introduction}
This is the installation manual of Tcl-Trf @mFullVersion@, as distributed
at @mDate@. For instructions on how to compile and install it see \file
{install.tex} or one of its derivatives (\file {install.dvi}, \file {install.ps}).

After torturing the world with 2 extensions releated to binary support
in tcl (\blob{} and \blobx{}) I could not resist and wrote another
one :-). It can be seen as a merger of features from these 2, but in
contrast to them it is build on top of the binary data support coming
in tcl8.0, aka

\begin{itemize}
\item	the \cmd{binio} command,
\item	memory channels,
\item	and transformation procedures.
\end{itemize}

\strong {This is not true in the current implementation}. We need

\begin{itemize}
\item	plain tcl 7.6 or higher (and the appropriate patch-file from
	subdirectory \file {patches}) or

\item	a tcl patched with Jan Nijtmans (nijtmans@nici.kun.nl)
	plus-patches as of 12/05/1996 or later.

\item	libz (compression library, optional, some functionality depends on it)
\item	libdes (DES encryption library, optional, some functionality depends on it)
\item	memchan 1.0 or higher
\end{itemize}


\chapter {Acknowledgements}

\begin {itemize}\Large
\item	Guess who ? John Ousterhout (JohnOusterhout@sun.eng.com) for
	\tcl {} itself. 
\item	Jacob Levy (jyl@sun.eng.com)
\item	Bastiaan Harmsen for general friendship.
\item	Graham Stoney (greyham@research.canon.oz.au) for \cmd {c2man}.
\item	Jan Nijtmans (nijtmans@nici.kun.nl) for support, encouragement
	(transformation procedures as part of his plus-patches) and
	several patches enhancing the usability (load-on-demand of
	libz, \dots{}) of this package.
\end   {itemize}

Neither the encryption algorithms nor the message digests were
implemented by myself, they are all from other persons. See chapter
\ref {copyright} for more information.


\chapter {Danger ahead}

First some remarks regarding the work with binary data (and this
package) in general:

\begin {enumerate}
\item	Whatever you do, don't forget to switch the used channels into
	binary mode, i.e. execute ``fconfigure <channel> -translation
	binary''. Not doing this will produce all kind of strange
	errors, even on unix machines, as the default translation mode
	(``auto'') will not only convert CR LF into LF, but a single
	CR as well!

\item	Using transformation procedures (as in this package)
	introduces an additional level of buffering because they are
	permitted to buffer all characters they are unable to process
	immediately. This may mess up especially
	interprocess communication, a notoriously problematic field. A
	not so improbable scenario would be the addition of encryption
	to enhance security. Using a blockcipher here may lead to the
	following problem (a line-oriented protocol is assumed):

	\begin {itemize}
	\item	The sender produces a line and writes it into the channel,
		waiting for a response from the receiver afterward.
	\item	The channel system invokes the blockcipher to encrypt
		the line, but not all characters are encrypted due to
		the length of the line not being a multiple of the
		blocksize. Especially the EOL-character is neither
		encrypted, nor transmitted.
	\item	The receiver decrypts the blocks it got, then waits
		for the EOL character.
	\item	\strong {DEADLOCK}.
	\end   {itemize}

	The solution here would be the usage of a blockcipher in CFB
	mode and shift $=$ 1 or that of a stream cipher, as both of these
	would encrypt single characters without buffering.

	Data compression is another good candidate for such a situation.

	Because of this I decided to include the number of characters
	buffered by a certain transformation in its description.

\end   {enumerate}


\chapter {The available commands}

All commands are available in 2 modes of operation, ``immediate'' and
``attached''. The generic syntactical forms of these modes are:

\begin {itemize}
\item[immediate:]	\cmd {command} -option arg {value} \dots{} \arg {source} \arg {destination}

			In this mode the command reads data from the
			channel \arg {source}, transforms it according
			to its specification (and the configuration as
			defined by the options) and then writes the
			result into the \arg {destination} channel.

\item[attached:]	\cmd {command} -attach \arg {channel} -option \arg {value} \dots{}

			Here a transformation procedure is created,
			configured as defined by the options and then
			attached to the \arg {channel}. All data read
			from or written into the \arg {channel} will
			be transformed accordingly from now on.
\end   {itemize}


\section {Encryption}

The ciphers implemented here are

\begin {enumerate}
\item	DES 	(the dinosaur)
\item	IDEA	(used by PGP)
\item	Blowfish
\item	RC4
\end   {enumerate}

The commands to invoke them are identical to the names, converted to
all lowercase. Options common to them all:

\begin {itemize}
\item [-direction]	Allowed values are ``encrypt'', ``decrypt'' and all
			unique abbreviations. Specifies the operation
			to be executed in immediate mode or the
			operation executed upon write! if attached to
			a channel.
\item [-key]		Argument to this option has to be a readable
			channel. The user key for encryption and/or
			decryption is defined by the contents of this
			channel. The minimum and maximum number of
			bytes to be read are cipher dependent and will
			be described later.
\end   {itemize}

The first 3 ciphers are blockciphers, i.e. they do not encrypt single
characters, but blocks of 2 or more. Such ciphers can be operated in 4
different modes and therefore have additional options defined:

\begin {itemize}
\item[-mode]	The 4 possible argument values (and modes) are:
	\begin {itemize}
	\item[ecb]	$=$ \strong {e}lectronic \strong {c}ode\strong {b}ook.
	\item[cbc]	$=$ \strong {c}ipher \strong{b}lock \strong{c}haining.
	\item[cfb]	$=$ \strong {c}ipher \strong {f}eed\strong {b}ack.
	\item[ofb]	$=$ \strong {o}utput \strong {f}eed\strong {b}ack.
	\end   {itemize}

	Only ``ecb'' neither requires a shiftvalue nor an initialization
	vector (see options below). All other (``stream'') modes require
	the initialization vector, the feedback modes additionally
	need a shiftvalue.

	The number of characters buffered internally depends both on
	cipher and operation mode.
	In 'ebc'- and 'cbc'-modes blocksize-1 characters are buffered.
	The feedback modes buffer shift-1 characters.


\item[-iv]	Argument to this option has to be a readable
		channel. The initialization vector required by the
		stream modes is defined by the contents of this
		channel. A complete block must be stored.

\item[-shift]	Only integer numbers are allowed as values to this
		option. It defines the number of bytes the feedback
		register is shifted to the left after every
		encryption. Only positive values not larger than the
		blocksize are allowed. They must be a divisor of the
		blocksize too!
\end   {itemize}


Cipher characteristics:

\begin {center}
	\begin {tabular}{|l|l|l|}\hline
	Name		& Blocksize (in byte)	& Keysize (in byte) \\ \hline\hline
	DES		& 8			& 8 (without MSB!) \\ \hline
	IDEA		& 8			& 16 \\ \hline
	Blowfish	& 8			& 1 - 56 \\ \hline\hline
	RC4		& -			& 1 - 256 \\ \hline
	\end   {tabular}
\end   {center}


\section {Message digests}

6 message digests are implemented:

\begin {enumerate}
\item	2 crc checkers, taken from PGP and library \lib {zlib}, under
	the names \cmd {crc} and \cmd {crc-zlib} resp.
\item	The adler-checksum, taken from \lib {zlib} too.
\item	MD5
\item	SHA (or SHS)
\item	HAVAL
\end   {enumerate}

The commands to invoke them are identical to the names, converted to
all lowercase. In immediate mode no options are allowed. Attached
digests have to submodes available:

\begin {itemize}
\item[-mode]	The allowed arguments are ``absorb'', ``write'' and
		all unique abbreviations.

		In the first submode written data is not changed
		except for attachment of the message digest at then
		end of the input. During reading a digest is computed
		too and then compared to the one attached. The
		'-matchflag' must be defined to propagate the result
		of the comparison to the tcl-level (``ok'' or ``failed'').

		In the second submode read or written data goes into
		nirwana, their digests are computed as
		sideeffect. After the channel is closed the computed
		digests are written to the channels specified with
		'-read/write-dest'.

\item[-matchflag]	The value specifies the name of global
			variable to write the result of a digest
			comparison to. Allowed for mode ``absorb''
			only. Its definition for a writeonly channel
			will be ignored.

\item[-write-dest]	The value must specify a writable channel. The
			message digest computed for data written to
			the attached channel is written into it after
			closing the attached channel. Allowed for mode
			``write'' only. Its definition for a readonly
			channel is ignored.

\item[-read-dest]	The value must specify a writable channel. The
			message digest computed for data read from
			the attached channel is written into it after
			closing the attached channel. Allowed for mode
			``write'' only. Its definition for a writeonly
			channel is ignored.
\end   {itemize}


\section {Data conversion}

All data conversion commands support only one option, ``-mode''.
Allowed values are ``encode'', ``decode'' and all unique
abbreviations. It specifies the operation to be executed in immediate
mode or the operation executed upon write! if attached to a channel.

Encode converts from binary data into the described representation,
decode does the reverse (obviously :-).

Standard conversions are
\begin {itemize}
\item[bin]	Converts bytes into dual representation.
\item[oct]	Converts bytes into octal representation.
\item[hex]	Converts bytes into hexadecimal representation.
\end   {itemize}
None of them buffers characters internally.

\begin {itemize}
\item[uuencode]	Blocks of 3 characters are uuencoded/decoded. Will
		buffer 2 characters internally until a block is complete.
\item[base64]	Blocks of 3 characters are
		base64-encoded/decoded. This is equivalent to PGP's
		ascii armor. Used by MIME too. Will buffer 2
		characters internally until a block is complete.
\item[ascii85]	Blocks of 4 characters will be encoded into Ascii85
		representation as defined in the 'Postscript Reference
		Manual' (2nd Edition, section 3.13, page 129).
\end   {itemize}
All of these are able to convert an incomplete block at the end of
input.

\section {Misc.}

4 other commands are available, not fitting into the categories above:

\begin	{enumerate}
\item	The reed-solomon error correcting coder available as \cmd
	{rs\_ecc}. The one and only option understood is ``-mode''. It
	is equivalent to the option understood by the conversion
	commands in the last section. The coder operates on blocks of
	248 characters each, therefore buffering 247 characters.

\item	A compressor based upon the \lib {zlib} library. It is
	available as \cmd {zip} and understands the following options:

	\begin	{itemize}
	\item[-mode]	Allowed arguments are ``compress'',
			``'decompress' and all unique abbreviations of
			these.
	\item[-level]	Possible arguments are ``default'' and integer
			numbers in the range 1 \dots{} 9. It specifies
			the used compression level.
	\end	{itemize}

\item	\cmd {unstack}. This command must be used to remove a
	conversion channel. The underlying channel is uncovered and
	accessible again. It requires a channelhandle as its only
	argument. Its application to a normal channel (file, socket,
	\dots {}) is identical to \cmd {close}.

\item	\cmd {binio}. See file \file {doc/binio.proposal} for more
	information.
\end	{enumerate}

\chapter {Copyrights}\label {copyright}

Copyright notes and acknowledgements for the different ciphers,
message digests, \dots {}:

\begin {enumerate}
\item	MD5 is copyright 'RSA Data Security, Inc'. The note at the
	beginning of \file {md5.h} says:

	\begin {quotation}
		License to copy and use this software is granted
		provided that it is identified as the "RSA Data
		Security, Inc. MD5 Message-Digest Algorithm" in all
		material mentioning or referencing this software or
		this function.
                                                                  
		License is also granted to make and use derivative
		works provided that such works are identified as
		"derived from the RSA Data Security, Inc. MD5
		Message-Digest Algorithm" in all material mentioning
		or referencing the derived work.
	\end   {quotation}

	I consider the terms of the license fullfilled by this
	quotation. 

\item	HAVAL is copyright $C^3SR$ and may not be sold or used as
	inducement to buy a product without the written permission of
	$C^3SR$. For technical questions or questions regarding the
	code used here contact the author of HAVAL (and code): Yuliang
	Zheng (yuliang@cs.uow.edu.au).

	I made some minor changes to eliminate masses of \cmd
	{gcc}-warnings regarding operator priority in expressions, but
	did not change the semantics.


\item	SHA / SHS was specified by the US government agency
	NIST. Primary author of the code used here seems to be Peter
	C. Gutmann (pgut01@cs.auckland.ac.nz), with heavily
	modifications from Uwe Hollerbach (uh@alumni.caltech.edu).

\item	The implementation of DES (libdes) used here is copyrighted by
	Eric Young (eay@mincom.oz.au).

\item	IDEA (International Data Encryption Algorithm) is a block
	cipher developed  by  Dr.  Xuejia  Lai  and
	Prof. Dr. J. L. Massey at the Swiss Federal Institute of
	Technology.

	The implementation used here was written by Richard De Moliner
	(demoliner@isi.ee.ethz.ch).

	\strong {Warning}:
	\begin {quotation}
		The IDEA(tm) block cipher is covered by a patent held
		by ETH and a Swiss company called Ascom-Tech AG.  The
		Swiss patent number is PCT/CH91/00117.  International
		patents are pending. IDEA(tm) is a trademark of
		Ascom-Tech AG.  \strong {There is no license fee required for
		noncommercial use}.  Commercial users may obtain
		licensing details from Dieter Profos, Ascom Tech AG,
		Solothurn Lab, Postfach 151, 4502 Solothurn,
		Switzerland, Tel +41 65 242885, Fax +41 65 235761. 
	\end {quotation}

\item	Blowfish was invented by Bruce Schneier
	(schneier@counterpane.com) and is decribed in the 2nd edition
	of his book ``Applied Cryptography''. His sources were used,
	with small modifications to interface it with this package.

	The statement of the author about the legal state of the
	algorithm:

	\begin {quotation}
		Blowfish is unpatented, and will remain so in all
		countries. The algorithm is hereby placed in the
		public domain, and can be freely used by anyone.
	\end {quotation}

\item	RC4 was invented by RSA Data Security, Inc and considered a
	``Trade Secret''. Somewhere in this big world it was
	reengineered and the result made public, rendering the secret
	useless.

	I don't know that RSADSI will try to inflict on people using it
	in the USA. Use it at your own risk.

\item	The Reed-Solomon error correcting coder is copyrighted by Paul
	Flaherty (paulf@stanford.edu) under the terms of the GNU
	General Public License.

\item	The library \lib {zlib} (base of compressor) is copyrighted by
	Jean-Loup Gailly (gzip@prep.ai.mit.edu) and Mark Adler
	(madler@alumni.caltech.edu).

\end   {enumerate}


\chapter {Future}

Things to be done in future.

\begin {itemize}
\item	The tests are incomplete, expand them.
\item	Add more ciphers (PGP!).
%\item	Add data compression.
\item	Transparent mode for digests.
\item	Channel doing special framing after a fixed (user-defined)
	number of characters
\item	Random-byte generator (as real channeltype, no transformation!)
\item	Identity-transformation with ability to gather statistics
	about the information flowing through it.
\item	Anybody out there wishing more ? Feel free to send me
	suggestions (\url {a.kupries@westend.com}).
\end   {itemize}

\end {document}
