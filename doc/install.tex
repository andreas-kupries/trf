% -*- latex -*-
\documentstyle {report}
\def\strong#1{{\bf #1}}
\def\lang#1{{\tt #1}}
\def\file#1{{\tt #1}}
\def\arg#1{{\tt $<$#1$>$}}
\def\trf{\strong {trf}}
\def\tcl{\lang {tcl}}
\def\tclsh{\cmd {tclsh}}
\def\wish{\cmd {wish}}
\def\cmd#1{{\tt #1}}
\def\opt#1{{\tt #1}}
\def\var#1{{\tt #1}}
\def\man#1{{\tt #1}}
\def\lib#1{{\tt #1}}
\def\url#1{{\tt #1}}
\def\tm{$^{\mbox{{\tiny (tm)}}}$}
\def\ra{$\rightarrow$}
\def\eq{$\equiv$}
\def\dos{{\tt DOS}}
\def\win{{\tt Windows\tm{}}}
\def\winnt{{\tt Windows NT\tm{}}}

\def\version{1.0}
\begin {document}
\tableofcontents

\chapter {Introduction}
This is the installation manual of Tcl-Trf @mFullVersion@, as of
@mDate@. For a documentation of the available commands see \file
{man.tex} or one of its derivatives (\file {man.dvi}, \file
{man.ps}).


\chapter {Common info}

Before starting compilation and installation a few notes:

\begin{itemize}
\item	You have managed to unpack this package in a directory of
	your choice, wether it was gzipped tar-archive or a
	zip-archive. The path of the file you are reading now is
	something like \file {\dots{}/trf@mFullVersion@/doc/install.?}.
	The \file {\dots{}/trf@mFullVersion@}-part of it is that we will
	call the \strong {source-directory} from now on.
\end{itemize}

This package depends in its operation on quite a few others. Here's
are list of them:

\begin{itemize}
\item	\strong {Tcl}, of course. The package will compile with 7.6
	and 8.0a2. Version 8.0a1 cannot be used anymore due to the
	differences in the object API for commands.

	This package requires manipulation of the core, hence a
	\strong {source distribution} is required. Instruction on how
	to patch it is provided in a separate chapter (\ra {} \ref {scalpel}).

\item	\strong {Memchan}. Trf can be used without, but the
	testsuite coming with this distribution requires it.

	Current version is 1.2. A binary distribution for Win32 is
	available.

	\begin {tabular} {|l|l|} \hline
	src & \url {ftp://ftp.westend.com/pub/aku/memchan1.2.tar.gz} \\ \hline
	src (win) & \url {ftp://ftp.westend.com/pub/aku/memchan12.zip} \\ \hline
	bin (win32) & \url {ftp://ftp.westend.com/pub/aku/mchan12b.zip} \\ \hline
	\end {tabular}
	
\item	\strong {zlib}. This package is used to implement the 'zip'
	compression command/filter. It will be loaded at runtime, if
	required. It is therefore optional. See chapter \ref {zlib} for
	more info.

\item	\strong {libdes}. This package is used to implement the 'des'
	encryption facility. It will be loaded at runtime, if
	required. It is therefore optional. See chapter \ref {libdes}
	for more info.

\item	\strong {SSLeay}. Another implementation of 'des',
	additionally used to implement 'rc2', 'md2' and 'sha-1'. Has
	priority over 'libdes' because of this. It will be loaded at
	runtime, if required. It is therefore optional. See chapter
	\ref {libdes} for more info.
\end{itemize}


\chapter  {Patching the core}\label {scalpel}

There is nothing to do if you are using a \tclsh{} or \wish{}
containing the plus-patches (as provided by Jan Nijtmans (\url {nijtmans@nici.kun.nl})).
For version 7.6plus please check that your patch was younger then Dec 12, 1996.
A plus-patch older than this does not contain the required modifications.
Skip ahead to the next section.


Well, it seems you have to do all the modifications by yourself. The
required files reside in the \file {patches}-subdirectory of the
source-directory. That is not quite true. They are in one of three
subdirectories, named after the version of tcl they belong to. Select
one of the directories and use the files in there. \strong {Don't mix
files from different subdirectories (\eq{} tcl versions)}.

Now you have two choices.

\begin {enumerate}
\item	Copy the 4 files \file {tcl.h}, \file {tclIO.c}, \file {configure.in}
	and \file {configure} to the appropriate locations, as given
	in the table below\footnote {People using \win{} should do
	this. They can safely ignore the files \file {configure} and \file
	{configure.in}, these are unix specific.} 

	\begin {tabular} {|l|l|} \hline
	tcl.h		& tcl$<$version$>$/generic/tcl.h	\\ \hline
	tclIO.c		& tcl$<$version$>$/generic/tclIO.c	\\ \hline
	configure.in	& tcl$<$version$>$/unix/configure.in	\\ \hline
	configure	& tcl$<$version$>$/unix/configure	\\ \hline
	\end {tabular}

	or,

\item	Apply the 2 patches \file {standard.patch} and \file
	{byteorder.patch}. The appropriate working directories are

	\begin {tabular} {|l|l|} \hline
	standard.patch	& tcl$<$version$>$/generic	\\ \hline
	byteorder.patch	& tcl$<$version$>$/		\\ \hline
	\end {tabular}
\end   {enumerate}


\chapter {zlib, zip compression library}\label{zlib}

\lib {zlib} is a freely available compression library using the
unpatented ``deflate''-algorithm employed by \cmd {gzip}, \cmd
{pkzip} and others. Current version is 1.0.4.

The official homepage is located at \url
{http://quest.jpl.nasa.gov/zlib/}. A downloading page is available
there too. Other sourcess of information are \file {rfc1950.txt},
\file {rfc1951.txt} and \file {rfc1952.txt}

\trf{} requires a shared library version of \lib {zlib}. People
without such \strong {don't} loose however.

\begin {itemize}
\item	The unix-specific makefile contains the target ``zlib''. This
	will generate a shared library for zlib. It assumes that the
	\lib {zlib} source distribution is available as subdirectory
	\file {zlib} (a sibling to \file {unix} !).

\item	A binary distribution for \win{} is available, see \url
	{http://www.winimage.com/zLibDll}. 

	A note of caution: The provided library kept crashing on my
	\winnt{} system. At least often enough to render it unusable in
	my eyes. YMMV however.

	Because of this the subdirectory \file {zlib.vc} contains
	everything to compile zlib as shared library (with MSVC++
	4.2). Just copy its contents into a zlib source distribution,
	copy \file {makefile.vc} to \file {makefile}, adapt the paths
	to your directory structure, then type \cmd {nmake}.

	At last copy the newly generated \file {zlib.dll} into the
	system directory of your machine.

\item	I have no information about its availability on Macintoshes.
\end   {itemize}




\chapter {libdes, DES encryption}\label {libdes}

\cmd {DES} support was originally based upon \lib{libdes}, a freely available
implementation of this algorithm, written by Eric A. Young (\url
{eay@mincom.com}). I used version 3.23 directly.

Higher versions are now part of \lib {SSLeay}, Erics implementation of
the \strong{S}ecure \strong{S}ockets \strong{L}ayer. Because of this I
have revamped the \cmd{des} loader code to use \lib {libcrypto.so} or
\lib {libdes.so}, in this order. This had the additional benefit of
being able to glue in even more crypto algorithms. These are
\cmd{md2}, \cmd{sha1} and \cmd{rc2}.


The official FAQ page is \url {http://www.psy.uq.edu.au:8080/~ftp/Crypto/}. 

Master locations are\vspace {1em}

{\small \begin {tabular} {|l|l|}\hline
     ftp://ftp.psy.uq.oz.au/pub/Crypto/SSL & SSLeay source \\
     ftp://ftp.psy.uq.oz.au/pub/Crypto/SSLapps & SSL applications \\
     ftp://ftp.psy.uq.oz.au/pub/Crypto/SSLapps/PORT4-5 & porting notes \\
     http://www.psy.uq.oz.au/~ftp/Crypto/ssl.html & Programmer Reference \\ \hline
\end {tabular}}\vspace {1em}

The list of mirror sites is too long to be included here, but the FAQ
contains them all.


\trf{} requires a shared library version of \lib {libdes}. People
without such \strong {don't} loose however.

\begin {itemize}
\item	The unix-specific makefile contains the target ``libdes''. This
	will generate a shared library for \lib {libdes}. It assumes that the
	\lib {libdes} source distribution is available as subdirectory
	\file {libdes} (a sibling to \file {unix} !).

\item	A binary distribution for \win{} is available, at the master
	location and the associated mirrors. This is not a binary
	distribution of \lib {libdes} alone, but of \lib {SSLeay}. The
	shared library \file {crypto32.dll} contains the required DES
	functionality\footnote {IDEA and RC4 as well, but these are
	\trf {} builtins. I am tempted to integrate the RSA
	functionality though}.

	Either rename this DLL to \file {libdes.dll} or uncomment the
	definition of \var {LIBDES} in \file {makefile.vc} before
	compilation of \trf {}.

\item	I have no information about its availability on Macintoshes.
\end   {itemize}

I'm sorry, but \lib {SSLeay} was to big to be included in the
Makefile. Instructions on how make a shared library for linux, sun,
solaris and irix are part of it however (as of version 0.6.6).



\chapter {Compilation and installation}
\section {Macintosh}

	This is a short section (therefore the first).

	I don't have a Mac, nor do I have access to one. Because
	of this I am unable to provide a Mac port of this
	extension.  I will most happily integrate any required
	changes, makefiles, et cetera into the distribution,
	if they are send to me (\url {a.kupries@westend.com}).


\section {Unix}

There are several different flavors of Un*x out there in this small
world. A \strong {configure} script, as generated by \strong {GNU
autoconf}, is provided to handle them all together.

\begin {itemize}
\item	Simply type:

\begin{verbatim}
	> configure
	> make
\end{verbatim}

The makefile will then check number and names of the files in the
package, generate dependency information, compile them and at last
build the library.

\item	For a shared library add the \opt {--enable-shared} to \cmd
	{configure}. 

\item	To check the extension, type \cmd {make test} or \cmd
	{make check}. This will build a tclsh containing just tcl and
	trf and then runs some test-scripts checking the behaviour of
	the created library. 

\item	At last type \cmd {make install} to install the generated
	library and the exported header. This installation conforms to
	the rules set by the package loader mechanism of the tcl-core. 
\end {itemize}

You do have a non-standard setup (tcl not installed at \file
{/usr/local}) ?

Don't fret, there are some tuning options for \cmd {configure} to help
you. 

\begin {itemize}
\item	The configure script searches the core header \file {tcl.h} in
	the following places:

	\begin {enumerate}
	\item	\$(TCL\_INCLUDE\_DIR)
	\item	\file {../../tcl8.0/generic}
	\item	\file {../../tcl8.0b*/generic}
	\item	\file {../../tcl8.0a2/generic}
	\item	\file {../../tcl7.6/generic}
	\item	\file {../../tcl/generic}
	\item	\file {../../tcl}
	\item	\$(prefix)
	\item	\file {/usr/local/include}
	\item	\file {/usr/include}
	\end {enumerate}

	By default \$(TCL\_INCLUDE\_DIR) is empty and \$(prefix) points
	to the location of an installed 'tclsh' or is \file {/usr/local} if
	tclsh could not be found via \$PATH.

	There are three options to overide this defaults:

	\begin {itemize}
	\item[]	\opt {--prefix}=DIR		   \ra {} \$(prefix)=DIR
	\item[]	\opt {--with-tcl}=DIR		   \ra {} \$(TCL\_INCLUDE\_DIR)=DIR/include
	\item[]	\opt {--with-tcl-include-dir}=DIR  \ra {} \$(TCL\_INCLUDE\_DIR)=DIR
	\end {itemize}


\item	To find \file {libtcl.a} (or \file {libtcl.so.??}) the paths

	\begin {enumerate}
	\item	\$(TCL\_LIB\_DIR)
	\item	\file {../../tcl8.0/unix}
	\item	\file {../../tcl8.0b*/unix}
	\item	\file {../../tcl8.0a2/unix}
	\item	\file {../../tcl7.6/unix}
	\item	\file {../../tcl/unix}
	\item	\file {../../tcl}
	\item	\$(exec\_prefix)
	\item	\file {/usr/local/include}
	\item	\file {/usr/include}
	\end {enumerate}

	are searched. \$(TCL\_LIB\_DIR) is empty by default, but
	can be set via

	\begin {itemize}
	\item[]	\opt {--with-tcl}=DIR	      \ra {} \$(TCL\_LIB\_DIR)=DIR/lib
	\item[]	\opt {--with-tcl-lib-dir}=DIR \ra {} \$(TCL\_LIB\_DIR)=DIR
	\end {itemize}

	\$(exec\_prefix) contains the same value as \$(prefix) by
	default, but can be set explicitly via \opt {--exec-prefix}.
\end {itemize}



\section {\win{}}

A single file \file {makefile.vc} is provided. It is designed to
work with \cmd {MSVC++ 4.2}. Just rename it to \file {makefile}, adapt
the paths\footnote {The available variables are described in the
makefile} at the top of it as required by your setup (aka directory
structure), then type\footnote {Inside of a \dos {} box}

\begin{verbatim}
	>	nmake
	>	nmake install
\end{verbatim}


The installation conforms to the rules set by the package loader
mechanism of the tcl-core.

Use \cmd {nmake test} to execute a selftest of the generated library.

\end {document}
